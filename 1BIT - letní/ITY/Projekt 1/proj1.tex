\documentclass[11pt]{article}
\usepackage[utf8]{inputenc}
\usepackage[czech]{babel}
\usepackage{times}
\usepackage{verbatim}
\usepackage{xspace}
\usepackage{multicol}
\usepackage{geometry}
 \geometry{
 a4paper,
 total={170mm,240mm},
 left=20mm,
 top=25mm,
 }

\begin{document}


	\title{Typografie a publikování \\ 1. projekt}
	\author{Matěj Mitaš \\ xmitas02@stud.fit.vutbr.cz \\}
	\date{}

	\maketitle

	\begin{multicols*}{2}
		\section{Hladká sazba}

			Hladká sazba je sazba z~jednoho stupně, druhu a řezu pí­sma sázená na stanovenou
			šířku odstavce. Skládá se z~odstavců, které obvykle začínají­
			zarážkou, ale mohou být sázeny i bez zarážky – rozhodují­cí­ je celková grafická
			úprava. Odstavce jsou ukončeny východovou řádkou. Věty nesmějí
			začínat číslicí.

			Barevné zvýraznění­, podtrhávání­ slov či různé velikosti písma vybraných slov se zde
			také nepoužívá. Hladká sazba je určena především pro delší­ texty, jako je napří­klad
			beletrie. Porušení­ konzistence sazby působí v~textu rušivě a~unavuje čtenářův zrak.

		\section{Smíšená sazba}

			Smíšená sazba má o~něco volnější­ pravidla než hladká sazba. Nejčastěji se klasická
			hladká sazba doplňuje o~další řezy pí­sma pro zvýraznění­
			důležitých pojmů. Existuje \uv{pravidlo}:

			\begin{quote}
				\hspace{5mm} Čí­m ví­ce \textbf{druhů}, \textbf{\textit{řezů}}, {\tiny
				velikostí}, barev pí­sma a jiných efektů použijeme, tí­ \textit{profesionálněji} bude
				dokument vypadat. Čtenář tím bude vždy {\Huge nadšen!}\hspace{10mm}
			\end{quote}


			\textsc{Tí­mto pravidlem se \underline{nikdy} nesmí­te ří­dit.} Příliš časté zvýrazňování
			textových elementů  a změny velikosti {\tiny pí­sma} jsou {\huge známkou} \textbf{\Huge
			amatérismu} autora a působí­ \textit{\textbf{velmi} rušivě}. Dobře navržený dokument nemá
			obsahovat ví­ce než 4 řezy či druhy pí­sma. \texttt{Dobře navržený dokument je decentní­, ne
			chaotický.}

			Důležitým znakem správně vysázeného dokumentu je konzistentní použí­vání­ různých druhů
			zvýraznění­. To napří­klad může znamenat, že \textbf{tučný řez} pí­sma bude vyhrazen pouze
			pro klíčová slova, \textit{skloněný řez} pouze pro doposud neznámé pojmy a nebude se to
			míchat. Skloněný řez nepůsobí­ tak rušivě a použí­vá se častěji. V~\LaTeX u jej sází­me
			raději pří­kazem \verb|\emph{text}| než \verb|\textit{text}.|

			Smíšená sazba se nejčastěji používá pro sazbu vědeckých článků a technických zpráv.
U~delší­ch dokumentů vědeckého či technického charakteru je zvykem upozornit čtenáře na
			význam různých typů zvýraznění­ v~úvodní­ kapitole.

		\section{České odlišnosti}

			Česká sazba se oproti okolní­mu světu v~některých aspektech mí­rně liší­. Jednou z~odlišností
			je sazba uvozovek. Uvozovky se v~češtině použí­vají­ převážně pro zobrazení­ pří­mé řeči.
V~menší­ míře se použí­vají­ také pro zvýraznění­ přezdí­vek a ironie. V~češtině se použí­vá tento
			\textbf{\uv{typ uvozovek}} namí­sto anglických ‘‘uvozovek‘‘. Lze je sázet připravenými
			příkazy nebo při použití UTF-8 kódování i přímo.

			Ve smíšené sazbě se řez uvozovek ří­dí­ řezem první­ho uvozovaného slova. Pokud je uvozována
			celá věta, sází­ se koncová tečka před uvozovku, pokud se uvozuje slovo nebo část věty,
			sází­ se tečka za uvozovku.

			Druhou odlišností je pravidlo pro sázení­ konců řádků. V~české sazbě by řádek neměl končit
			osamocenou jednopí­smennou předložkou nebo spojkou.
			Spojkou \uv{a} končit může při sazbě do 25 liter. Abychom LaTeXu zabránili v~sázení
			osamocených předložek, vkládáme mezi předložku a slovo \textbf{nezlomitelnou mezeru}
			znakem \char`\~ (vlnka, tilda). Pro automatické doplnění vlnek slouží volně šiřitelný 
			program \textit{vlna} od pana Olšáka\footnote{Viz http://petr.olsak.net/ftp/olsak/vlna/.}.
			
			

	\end{multicols*}
\end{document}