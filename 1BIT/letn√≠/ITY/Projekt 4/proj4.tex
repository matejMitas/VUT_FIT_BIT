\documentclass[11pt]{article}
\usepackage[czech]{babel}
\usepackage[utf8]{inputenc}
\usepackage{times}
\usepackage{verbatim}
\usepackage{xspace}
\usepackage{multicol}
\usepackage{geometry}

\usepackage[square,sort,comma,numbers]{natbib}

\usepackage{url}
\DeclareUrlCommand\url{\def\UrlLeft{<}\def\UrlRight{>} \urlstyle{tt}}


\geometry{
 a4paper,
 total={170mm,240mm},
 left=20mm,
 top=30mm,
}
 

\begin{document}

\begin{titlepage}
	\begin{center}
    	\Huge
		\textsc{\Huge{Vysoké učení technické v~Brně}\\ \huge{Fakulta informačních technologií}} \\
		\vspace{\stretch{0.35}}
		\LARGE Typografie a publikování - 4.projekt\\
		\Huge{Citace}
    	\vspace{\stretch{0.65}}
	\end{center}
	{\Large \today \hfill Matěj Mitaš}
\end{titlepage}

\section*{Co je typografie?}

Typografie je sama o sobě široký pojem. Někteří mohou poukazovat na formální stránku, zabývat se věcmi čistě pragmatickými a exaktními jako je teorie čitelnosti nebo řemeslné zpracování jednotlivých glyfů. Do rukou těchto jedinců patří světoznámá a kritiky uznávaná kniha \emph{\uv{Thinking with Type}}\cite{kniha_1}. Někdy se může stát nepříjemná věc přílišného zaměření na formální stránku věci, když ve skutečnosti je základ myšlenka, kterou chceme předat \cite{mag_clanek_2}. Pokud ovšem toužíte po pravidelnějších dávkách typografické úžasnosti je možné odebírat velice známý typografický časopis \emph{\uv{Baseline}} \cite{magazin}.

Jenže jako každá strana má dvě mince, i typografie se vyznačuje mnoha způsoby chápání. Někdo ji chápe také jako formu projevu \cite{prace_1}. Chtělo by se dokonce říci uměleckou disciplínu. V tomto článku \cite{online_3} jsou probírány základy kaligrafie, tj. krasopisu. Avšak i pozitivní či neutrální věci mohou být zneužity. Problematika používání typografie jako prostředku síly je prodiskutována zde \cite{mag_clanek_1}.

\section*{Typografie na webu}

V našich životech vždy hrála typografie vždy velmi významnou roli. S přesunem majoritní části naší pozornosti k obrazovkám osobním počítačů a jejich derivátů stoupá potřeba se zorientovat na tom novém poli \cite{kniha_2}. Jednou z nejdůležitějších polí aplikace typografie na obrazovách je web \cite{online_2} . Proto je potřeba vzít v potaz všechny možné vzniknuvší aspekty: \emph{\uv{Na velkém monitoru a maximalisovaném okně prohlížeče můze vzniknou na jednom řádku cca 300 znaků.}} \cite{online_1}. Není ale možno se pouze soustředit na technické provedení dané věci bez znalosti základních principů fyzické implementace a tvorby samotných fontových rodin \cite{prace_2}.


\newpage


\bibliographystyle{czplain}
\bibliography{proj4}

\end{document}