\documentclass[11pt]{article}
\usepackage[utf8]{inputenc}
\usepackage[czech]{babel}
\usepackage{times}
\usepackage{verbatim}
\usepackage{xspace}
\usepackage{multicol}
\usepackage{geometry}
\usepackage{amsthm}
\usepackage{amssymb}
\usepackage{amsmath}
 \geometry{
 a4paper,
 total={180mm,250mm},
 left=15mm,
 top=25mm,
 }
 
% ================= Definitions of theorems
\theoremstyle{definition}  
\newtheorem{def1}{Definice}[section]
\theoremstyle{plain}
\newtheorem{def2}[def1]{Algoritmus}
\newtheorem{def3}{Věta}
% ================= Definitions




\begin{document}

	\begin{center}
    	\Huge
		\textsc{Fakulta informačních technologií \\Vysoké učení technické v Brně} \\[84mm]
		\LARGE Typografie a publikování - 2.projekt\\
		Sazba dokumentů a matematických výrazů
		\\[122mm]
    	\vfill
	\end{center}
	{\LARGE 2017 \hfill Matěj Mitaš}
	

	%\maketitle
	
	%FAKULTA INFORMAČNÍCH TECHNOLOGIÍ
	%VYSOKÉ UČENÍ TECHNICKÉ V BRNĚ
	
	%Typografie a publikování - 2. projekt
	%Sazba dokumentů a matematických výrazů
	
	%\mathrm odstranit z rovnice
	\thispagestyle{empty}
	

	\begin{multicols*}{2}	
	
	\section*{Úvod}
	V této úloze si vyzkoušíme sazbu titulní strany, matematických vzorců, prostředí
	dalších textových struktur obvyklých pro technicky zaměřené texty (například
	rovnice \eqref{first} nebo definice \ref{test} na straně \pageref{test}).

	Na titulní straně je využito sázení nadpisu podle optického středu s využití
	zlatého řezu. Tento postup byl probírán na přednášce.


	\section{Matematický text}
	\label{first}

	Nejprve se podíváme na sázení matematických symbolů a výrazů v plynulém textu. Pro
	množinu $V$ označuje $card(V)$ kardinalitu $V$.
	Pro množinu $V$ reprezentuje $V^*$ volný monoid generovaný množinou $V$ s operací
	konkatenace. Prvek identity ve volném monoidu $V^*$ značíme symbolem $\epsilon$.
	Nechť $V^+ = V^* - \{\epsilon\}$. Algebraicky je tedy $V^+$ volná pologrupa
	generovaná množinou $V$ s operací konkatenace. Konečnou neprázdnou množinu $V$
	nazvěme abeceda. Pro $\omega \in V^*$ označuje $|\omega|$ délku řetězce $\omega$.
	Pro $W \subseteq C$ označuje $occur(\omega, W)$ počet výskytů symbolů z $W$ v
	řetězci $\omega$ a $sym(\omega, i)$ určuje $i$-tý symbol řetězce $\omega$;
	například $sym(abcd, 3) = c$.
	Nyní zkusíme sazbu definic a vět s využitím balíku \texttt{amsthm}.
	
	\begin{def1}
	\label{test}
	Bezkontextová gramatika je čtveřice $G = (V, T, P, S)$, kde $V$ je totální abeceda, $T
	\subseteq V$ je abeceda terminálů, $ S \in (V-T)$ je startující symbol a $P$ je konečná množina pravidel tvaru $ q: A \alpha \in P$,
	kde $ A \in ( V - T ), \alpha \in V^* $ a $q$ je návěští tohoto pravidla. Nechť $N = V - T$ značí abecedu neterminálů.
	
	Pokud $ A \alpha \in P$ , $ \in V^*$, $G$ provádí derivační krok z $\gamma A \delta$ do $\gamma \alpha \delta$ podle pravidla $[q\colon A\rightarrow \alpha] $, symbolicky
	píšeme $\gamma A \delta \Rightarrow \gamma \alpha \delta [q\colon A\rightarrow \alpha] $ nebo zjednodušeně $\gamma A \delta \Rightarrow \gamma \alpha \delta$. Standardním způsobem definujeme $\Rightarrow^m$, kde $m\geq 0$ . Dále
	definujeme tranzitivní uzávěr  $\Rightarrow^+$ a tranzitivně-reflexivní uzávěr  $\Rightarrow^*$ .
	
		Algoritmus můžeme uvádět podobně jako definice textově, nebo využít pseudokódu vysázeného ve vhodném prostředí (například \texttt{algorithm2e}).
	\end{def1}	

	\begin{def2}Algoritmus: Algoritmus pro ověření bezkontextovosti gramatiky. Mějme gramatiku G = (N, T, P, S).
	\begin{enumerate}
		\item Pro každé pravidlo $p \in P$ proveď test, zda $p$ na levé straně obsahuje právě jeden symbol z $N$.
		\item Pokud všechna pravidla splňují podmínku z kroku $1$, tak je gramatika $G$ bezkontextová.
	\end{enumerate}
	\end{def2}


	\begin{def1}
		Jazyk definovaný gramatikou $G$ definujeme jako $L(G) = \{ w \in T^* | S \Rightarrow^* w\}$ .
	\end{def1}

	\subsection{Podsekce obsahující větu}
	\begin{def1}
	 	Nechť $L$ je libovolný jazyk. $L$ je bezkontextový jazyk, když a jen když $L = L(G)$, kde $G$ je libovolná bezkontextová gramatika.
	\end{def1}
	
	\begin{def1}
		Množinu $\mathcal{L}_{CF} = \{L|L$ nazýváme třídou bezkontextových jazyků.
	\end{def1}
	
	\begin{def3}
		Nechť $L_{abc} = \{a^n b^n c^n | n \geq 0\}$ Platí, že $L_{abc} \notin \mathcal{L}_{CF}$.
	\end{def3}
	
Důkaz: Důkaz se provede pomocí Pumping lemma pro bezkontextové jazyky, kdy ukážeme, že není možné, aby platilo, což bude implikovat pravdivost věty ... .





	\section{Rovnice a odkazy}

Složitější matematické formulace sázíme mimo plynulý text. Lze umístit několik výrazů na jeden řádek, ale pak je třeba tyto vhodně oddělit, například příkazem \verb|\quad|. 

	$$
		\sqrt[x^2]{y_0^3} \quad \mathbb{N} = \{0, 1, 2, \dots\} \quad x^{y^y} \ne x^{yy} \quad 
		z_{i_j} \not\equiv z_{ij}
	$$

	V rovnici (...) jsou využity tři typy závorek s různou explicitně definovanou velikostí.

	\begin{eqnarray}
		\bigg\{\Big[\big(a + b\big) * c\Big]^d + 1\bigg\} = x\\
		\lim_{x\rightarrow\infty} \frac{sin^2x + cos^2x}{4} = y \nonumber
	\end{eqnarray}
	

	V této větě vidíme, jak vypadá implicitní vysázení limity $\lim_{x\rightarrow\infty} f(n)$ v normálním odstavci textu. Podobně je to i s dalšími symboly jako $\sum_{1}^{n}$ či $U_{A\in \mathcal{B}}$ . V případě vzorce $\lim\limits_{x\rightarrow 0} \frac{sinx}{x}$ jsme si vynutili méně úspornou sazbu příkazem \verb|\limits|.
	
	\begin{eqnarray}
		\int_{a}^{b} f(x) dx &=& -\int_{a}^{b} f(x)' dx\\\nonumber
		\Big(\sqrt[5]{x^4}\Big)' = \Big(x^{\frac{4}{5}}\Big)' &=& \frac{4}{5}x^{-\frac{1}{5}}  \frac{4}{5\sqrt[5]{x}}\\\nonumber
		\overline{\overline{A \vee B}} &=& \overline{\overline{A} \wedge \overline{B}\nonumber}
	\end{eqnarray}


	\section{Matice}

Pro sázení matic se velmi často používá prostředí \texttt{array} a závorky (\verb|\left, \right|). 
	
	\begin{equation}
		\begin{pmatrix}
		a + b & b + a\\
		\widehat{\xi + \omega} & \hat{\pi} \\
		\vec{a} & \overleftrightarrow{AC} \\
		0 & \beta\\
		\end{pmatrix}\nonumber
	\end{equation}
	
	\begin{equation}
		A = 
 \begin{Vmatrix}
  a_{1,1} & a_{1,2} & \cdots & a_{1,n} \\
  a_{2,1} & a_{2,2} & \cdots & a_{2,n} \\
  \vdots  & \vdots  & \ddots & \vdots  \\
  a_{m,1} & a_{m,2} & \cdots & a_{m,n} 
 \end{Vmatrix}\nonumber
	\end{equation}
	
	\begin{equation}
		\begin{vmatrix}
		t & u\\
		v & w\\
		\end{vmatrix} = tw - uv\nonumber
	\end{equation}


	
Prostředí \texttt{array} lze úspěšně využít i jinde.

	$$ \binom{n}{k} =
	\left\{
		\begin{array}{l l}
			\frac{n!}{k!(n-k)!} & \text{pro } 0 \leq k \leq n \\
			0 & \text{pro } k < 0 \text{ nebo } k > n
		\end{array}
	\right. $$

	\section{Závěrem}

V případě, že budete potřebovat vyjádřit matematickou konstrukci nebo symbol a nebude se Vám dařit jej nalézt v samotném \LaTeX u, doporučuji prostudovat možnosti balíku maker \AmS\LaTeX.
Analogická poučka platí obecně pro jakoukoli konstrukci v \TeX u.
	
	
	

	\end{multicols*}
\end{document}